\documentclass[]{article}
\usepackage{lmodern}
%\usepackage{amssymb}
\usepackage{amsmath}
\usepackage{ifxetex,ifluatex}
\usepackage{fixltx2e} % provides \textsubscript
\ifnum 0\ifxetex 1\fi\ifluatex 1\fi=0 % if pdftex
  \usepackage[T1]{fontenc}
  \usepackage[utf8]{inputenc}
\else % if luatex or xelatex
  \ifxetex
    \usepackage{mathspec}
    \usepackage{xltxtra,xunicode}
  \else
    \usepackage{fontspec}
  \fi
  \defaultfontfeatures{Mapping=tex-text,Scale=MatchLowercase}
  \newcommand{\euro}{€}
\fi
% use upquote if available, for straight quotes in verbatim environments
\IfFileExists{upquote.sty}{\usepackage{upquote}}{}
% use microtype if available
\IfFileExists{microtype.sty}{\usepackage{microtype}}{}
\usepackage[margin=1in]{geometry}
\ifxetex
  \usepackage[setpagesize=false, % page size defined by xetex
              unicode=false, % unicode breaks when used with xetex
              xetex]{hyperref}
\else
  \usepackage[unicode=true]{hyperref}
\fi
\hypersetup{breaklinks=true,
            bookmarks=true,
            pdfauthor={},
            pdftitle={},
            colorlinks=true,
            citecolor=blue,
            urlcolor=blue,
            linkcolor=magenta,
            pdfborder={0 0 0}}
\urlstyle{same}  % don't use monospace font for urls
\setlength{\parindent}{0pt}
\setlength{\parskip}{6pt plus 2pt minus 1pt}
\setlength{\emergencystretch}{3em}  % prevent overfull lines
\setcounter{secnumdepth}{0}

\date{}

\usepackage{fancyhdr}
\pagestyle{fancy}
\addtolength{\headheight}{\baselineskip}
\lhead{Chaim-Leib Halbert (SID 20516204)\\Saul Fuhrmann}
\chead{Homework \#1}
\rhead{BS"D\\EE 120, Fall 2014}
\lfoot{}
\cfoot{\thepage}
\rfoot{}
\renewcommand{\headrulewidth}{0.4pt}
\renewcommand{\footrulewidth}{0 pt}

\begin{document}

\begin{enumerate}
\def\labelenumi{\arabic{enumi})}
\item
  \begin{enumerate}
  \def\labelenumii{\alph{enumii}.}
  \item
    The impulse response of the system \(y(t) = x(t) + \alpha y(t - T)\)
    is simply: \[
        h(t) = \sum_{k = 0}^\infty \alpha^k \delta(t - kT)
    \]
  \item
    The system is BIBO stable if \(\int_infty^\infty |h(t)|dt\) is
    bound. In our case we get that this integral is equal: \[
        \int_\infty^\infty \left| 
            \sum_{k = 0}^\infty \alpha^k \delta(t - kT)
        \right| dt
    \]

    If \(\alpha\) is positive we get that: \[
        \sum_{k = 0}^\infty \int_0^\infty \alpha^k \delta(t - kT)dt  = 
            \sum_{k = 0}^\infty \alpha^k
    \]

    This is bound if \(\alpha < 1\) and unbound otherwise. Hence, if
    \(0 \leq \alpha < 1\) the system is BIBO stable, and it is unstable
    otherwise.
  \item
    \[
        h(t) = \sum_{k = 0}^\infty \alpha^k \delta(t - kT)
    \] can also be inverted with the LTI system \[
        \boxed{h_1(t) = \delta(t) - \alpha \delta(t - T)}
    \]

    We can show this by looking at the properties of the convolution:

    \begin{align*}
        y(t) &= x(t) * h(t) \\
        h_1(t) * y(t) &= h_1(t) * \Big( x(t) * h(t) \Big) \\
        &= \Big( h_1(t) * h(t) \Big) * x(t)
    \end{align*}

    Therefore, if \((h_1(t) * h(t)) = \delta(t)\) (which is trivial) we
    get that \[
        h_1(t) * y(t)  = x(t)
    \]
  \end{enumerate}
\item
  \begin{enumerate}
  \def\labelenumii{\alph{enumii}.}
  \item
    \textbf{{[}ADD Diagram{]}}
  \item
    \[
        \ddot{y}(t) + 300\dot{y}(t) + 2 \times 10^4 y(t) = 10^3 \dot{x}(t)
    \]

    Since \(e^{jwt}\) is an eigenfunction for the above LTI system, we
    get that

    \begin{align*}
        y(t) &= A_w e^{jwt} \\
        \dot{y}(t) &= A_w (jw) e^{jwt} \\
        \ddot{y}(t) &= A_w (wj)^2 e^{jwt} \\
        \dot{x}(t) &= (jw) e^{jwt}
    \end{align*}

    Substituting all of these into the equation yields:

    \begin{align*}
        A_w (jw)^2 e^{jwt} + A_w 300 (jw) e^{jwt} + A_w 2 \times 10^4 e^{jwt} &= 
            10^3 e^{jwt} \\
        e^{jwt} A_w \Big( (jw)^2 + 300(jw) + 10^4 \Big) &= e^{jwt} 10^3
    \end{align*}\begin{align*}
        A_w &= \frac{10^3}{-w^2 + 10^4 + 300 w j} \\
            &= \boxed{\frac{10^3 (-w^2 + 10^4 - 300wj)}{ (-w^2 + 10^4)^2 + (300 w)^2}}
    \end{align*}
  \end{enumerate}
\item
  \begin{enumerate}
  \def\labelenumii{\alph{enumii}.}
  \item
    \textbf{{[}ADD Diagram{]}}
  \item
    \[
        y[n] + 20 y[n - 1] + 1700 y[n - 2] = x[n] + 20x[n-1]
    \]

    Since \(e^{jwn}\) is an eigenfunction for the above LTI system, we
    get that

    \begin{align*}
        y[n] &= A_w e^{jwt} \\
        y[n - 1] &= A_w e^{jwn} e^{-jw} \\
        y[n - 2] &= A_w  e^{jwn} e^{-2jw} \\
        x[n] &= e^{jwn}
    \end{align*}

    Substituting all of these into the equation yields:

    \begin{align*}
        A_w e^{jwt} + A_w e^{jwt}20  e^{-jw} + A_w e^{jwt} 1700 e{-2jw} &= 
            e^{jwt} + e^{jwt} 20 e^{-jw} \\
        A_w e^{jwt}(1 +  20  e^{-jw} + 1700 e{-2jw}) &=
            e^{jwt}(1 + 20e^{-jw})
    \end{align*}

    \[
        \boxed{A_w = \frac{(1 + 20e^{-jw})}{(1 +  20  e^{-jw} + 1700 e{-2jw})}}
    \]
  \end{enumerate}
\item
  \begin{enumerate}
  \def\labelenumii{\alph{enumii}.}
  \item
    \(\Pi(t / 8) * comb(t / 10)\), by examination of the signal, we get
    that the period is \(T_0 = 10\) and the fundamental frequency
    \(\boxed{w_0 = \frac{\pi}{5}}\). We can compute \(a_k\) by:

    \[\begin{aligned}
        a_k = 
            10\int_{-4}^{4} e^{-jw_0 k t} dt = 
            10 \frac{1}{-jw_0kt} \left(  e^{-jw_0k4}- e^{jw_0k4}\right) \\
        = \boxed{ \frac{20Sin(4w_0 k)}{10 w_0 k} } 
    \end{aligned}\]
  \item
    \(\Pi(4t) * comb(t / 10)\), by examination of the signal, we get
    that the period is \(T_0 = 10\) and the fundamental frequency
    \(\boxed{w_0 = \frac{\pi}{5}}\). We can compute \(a_k\) by:

    \[\begin{aligned}
        a_k = 
            10 \int_{-\frac{1}{8}}^{\frac{1}{8}} e^{-jw_0 k t} dt  = 
            \frac{10}{-jw_0kt } \left( 
                e^{-jw_0k\frac{1}{8}}- e^{jw_0k\frac{1}{8}}
            \right) \\
        = \boxed{ \frac{20Sin(\frac{1}{8}w_0 k)}{w_0 k} }
    \end{aligned}\]
  \item
    \textbf{{[}TODO{]}}
  \end{enumerate}
\end{enumerate}

\begin{enumerate}
\def\labelenumi{\arabic{enumi}.}
\setcounter{enumi}{4}
\item
  \begin{enumerate}
  \def\labelenumii{\alph{enumii}.}
  \item
    \textbf{{[}TODO{]}}
  \item
    \textbf{{[}TODO{]}}
  \end{enumerate}
\item
  \begin{enumerate}
  \def\labelenumii{\alph{enumii}.}
  \item
    \textbf{{[}TODO{]}}
  \item
    \textbf{{[}TODO{]}}
  \item
    \textbf{{[}TODO{]}}
  \item
    \textbf{{[}TODO{]}}
  \end{enumerate}
\end{enumerate}

\end{document}
