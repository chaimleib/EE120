\documentclass[10pt, letterpaper]{article}
\usepackage{amsmath}
\usepackage[margin=1.25in]{geometry}
\usepackage{scrextend}
\usepackage{fancyhdr}

\pagestyle{fancy}
\addtolength{\headheight}{\baselineskip}
\lhead{Chaim-Leib Halbert (SID 20516204)\\Saul Fuhrmann}
\chead{Homework \#1}
\rhead{BS"D\\EE 120, Fall 2014}
\lfoot{}
\cfoot{\thepage}
\rfoot{}
\renewcommand{\headrulewidth}{0.4pt}
\renewcommand{\footrulewidth}{0 pt}

% indent
\newcommand{\ind}[2]{
	\begin{addmargin}[#1]{0pt}
		{#2}
	\end{addmargin}
}

\begin{document}
\subsection*{1}
\textbf{a.} Radiosurgery

\ind{4 em}{
\begin{description}
    \item[Input:] radiation beam
    \item[System:] patient
    \item[Output:] effect on patient
\end{description}

System is partially-known, output is directly measured, input is designed
}

\textbf{b.} medical magnetic resonance imaging

\ind{4 em}{
\begin{description}
    \item[Input:] magnetic beam
    \item[System:] patient
    \item[Output:] resonance measurements
\end{description}

System is unknown, output is directly measured, input is known
}

\textbf{c.} reflection seismology

\ind{4 em}{
\begin{description}
    \item[Input:] explosion
    \item[System:] earth
    \item[Output:] pressure wave measurements
\end{description}

System is unknown, output is directly measured, input is known
}

\textbf{d.}
\textbf{e.}
\subsection*{2}

\textbf{a.}
\textbf{b.}
\textbf{c.}
\textbf{d.}
\textbf{e.}
\textbf{f.}
\subsection*{3}
\textbf{a.}
$\delta_1(t) = \lim_{a \to \infty} \frac{a}{2} e^{-a|t|}$\\*
Property \#1 \\*
$\delta_1(t - t_0) = \lim_{a \to \infty} \frac{a}{2} e^{-a|t - t_0|}$

Let's explore two cases $|t - t_0| = 0$ and $|t - t_0| = m \neq 0$, in case 1, 
the limit takes the value of $\lim_{a \to \infty} \frac{a}{2} e^{0} = \lim_{a \to \infty} \frac{a}{2}$ 
which is clearly non-zero.

When $|t - t_0| = m \neq 0$ the limit is:
\[
\lim_{a \to \infty} \frac{a}{2} e^{-am} = 
\lim_{a \to \infty} \frac{\frac{a}{2} }{e^{am}}
\]
using L'Hopital's rule we get that:
\[ =
\lim_{a \to \infty} \frac{\frac{1}{2} }{me^{am}} = 0
\]
Property \#2
\[
\lim_{a \to \infty} \int_{-\infty}^{\infty}\frac{a}{2} e^{-a|t|}\,dt = 
\lim_{a \to \infty} \frac{a}{2} 2 \int_{0}^{\infty} e^{-at}\,dt 
\]
\[
= \lim_{a \to \infty} \frac{1}{a} e^{-at} |_{0}^{\infty}
= \lim_{a \to \infty} \frac{a}{a}
\]
\[= 1\]
\textbf{b.}
$\delta_2(t) = \lim_{a \to 0} \frac{1}{2a} \Pi(\frac{t - a}{2a})$\\*
Property \#1
\[
\delta_2(t - t_0) = \lim_{a \to 0} \frac{1}{2a} \Pi(\frac{(t -t_0) - a}{2a})
\]
By using standard shifting and scaling we can write $\Pi(\frac{(t -t_0) - a}{2a})$ piecewise:
\[\Pi(\frac{(t -t_0) - a}{2a}) = \left\{
  \begin{array}{lr}
    1 & : t \in [t_0, t_0 + 2a]\\
    0 & : t \not\in [t_0, t_0 + 2a]
  \end{array}
\right.
\]
As the $a \to 0$ the only value that stays consistantly non-0 is $t = t_0$
all the other values become zero.
\\*
Property \#2
\[
\lim_{a \to 0} \int_{-\infty}^{\infty}\frac{1}{2a} \Pi(\frac{t - a}{2a})dt = 
\lim_{a \to 0} \frac{1}{2a} \int_{0}^{2a}dt 
= \lim_{a \to 0} \frac{2a}{2a} 
\]
\[= 1\]
\subsection*{4}
\textbf{a.}
\[z \times z^* = (re^{j\theta})(re^{-j\theta}) = r^2 e^{j(\theta - \theta) = r^2} \]
\textbf{b.}
\[ \frac{z}{z^*} = \frac{re^{j\theta}}{re^{-j\theta}} = \frac{r}{r} e^{j (\theta + \theta)} = e^{2j\theta} \]
\textbf{c.}
\[(z_1 z_2)^* = (r_1 e^{j \theta_1} r_2 e^{j \theta_2})^* = (r_1 r_2 e^{-j(\theta_1 + \theta_2)})^* = r_1 r_2 e^{-j(\theta_1 + \theta_2)}\]
\[z_1^* z_2^* = (r_1e^{j\theta_1})^* (r_2e^{j\theta_2})^* = (r_1e^{-j\theta_1}) (r_2e^{-j\theta_2})  = r_1 r_2 e^{-j(\theta_1 + \theta_2)} \]
\[\rightarrow z_1^* z_2^* = (z_1 z_2)^*\]
\textbf{d.}
\[
(\frac{z_1}{z_2})^*  = (\frac{r_1e^{j\theta_1}}{r_2e^{j\theta_2}})^* =   (\frac{r_1}{r_2} e^{j(\theta_1 - \theta_2)})^* =     
\frac{r_1}{r_2} e^{j(\theta_2 - \theta_1)}
\]

\[
\frac{z_1^*}{z_2^*} =
\frac{r_1 e^{-j\theta_1}}{r_2 e^{-j\theta_2}} = 
\frac{r_1}{r_2} e^{j(\theta_2 - \theta_1)}
\]
\[
\rightarrow
\frac{z_1^*}{z_2^*} =(\frac{z_1}{z_2})^* 
\]

\subsection*{5}
\textbf{a.} $u(t)$
\[ Odd = \frac{u(t) - u(-t)}{2}\]
\[Evem = \frac{u(t) + u(-t)}{2} =  \frac{1}{2}\]
\textbf{b.} $Cos(2\pi t)u(t)$
\[Odd = \frac{Cos(2\pi t)u(t) - Cos(-2\pi t)u(-t)}{2} = \frac{Cos(2\pi t)}{2}(u(t) - u(-t))\]
\[Even = \frac{Cos(2\pi t)u(t) + Cos(-2\pi t)u(-t)}{2} = \frac{Cos(2\pi t)}{2}(u(t) + u(-t))\]
\[ = \frac{Cos(2\pi t)}{2}\]


\subsection*{6}
\textbf{a.} $h(t) = e^tu(t)$ is not \emph{BIBO} stable system, an example of a bounded input 
resulting in an unbound output is $x(t) = u(t)$ which results in the output
\begin{align*}
y(t) = h(t) * x(t) = 
\int_{-\infty}^{\infty} u(\tau)e^{t - \tau}u(t - \tau) \, d\tau \\
= \int_{0}^{t}e^{t - \tau} \, d\tau 
= e^t \int_{0}^{t}e^{-\tau} \, d\tau = \\
= \boxed{e^t (1 - e^{-t})}
\\
\end{align*}
which is an unbound function.

\textbf{b.} $h(t) = (t - 1)^2e^{1-t}u(t)$ The system is \emph{BIBO} stable since:
\begin{align*}
	y(t) = h(t) * x(t) = 
	\int_{-\infty}^{\infty} |u(t)e^{1 - t}(t - 1)^2 \, dt \\
	= e
\end{align*}
which is bound, hance the integral is bound.

\textbf{c.} $h[n] = u[n - 4]$ is not \emph{BIBO} stable system, an example of a bounded input 
resulting in an unbound output is $x(t) = u(t)$ which results in the output:
\begin{align*}
y[t] = h[t] * x[t] = 
\sum_{k = -\infty}^{\infty} u[k] u[n - k - 4] = \sum_{k = 0}^{n - 4} 1 \\
= n - 4 + 1 = \boxed{n - 3}
\end{align*}
which is an unbound function. Therefore, the system is not BIBO stable

\textbf{d.} $Cos[2\pi n] u[n]$ is not \emph{BIBO} stable system, an example of a bounded input 
resulting in an unbound output is $x[n]= u[n]$ which results in the output
\begin{align*}
y[t] = h[t] * x[t] = 
\sum_{k = -\infty}^{\infty} u[k] u[n - k] 
= \sum_{k = 0}^{n} 1 
= \boxed{n +1 }
\end{align*}
which is clearly unbound.

\textbf{e.} $\sum_{n = -\infty}^{\infty}\delta(t - 2n)$ is not \emph{BIBO} stable system, an example of a bounded input 
resulting in an unbound output is $x(t)= u(t)$ which results in the output
\begin{align*}
	y(t) = h(t) * x(t) = 
	\int_{-\infty}^{\infty} u(t - \tau)\sum_{n = -\infty}^{\infty} \delta(\tau - 2n) \, d\tau \\
	= \int_{-\infty}^{t} \sum_{n = -\infty}^{\infty} \delta(\tau - 2n) \, d\tau = 
	\sum_{n = -\infty}^{\infty}  \int_{-\infty}^{t} \delta(\tau - 2n) \, d\tau \\
	= \sum_{n = -\infty}^{\left \lfloor{\frac{t}{2}}\right \rfloor } 1 
	= \boxed{\infty}
\end{align*}
which is clearly unbound
\subsection*{7}
\textbf{a.} 
$x(t) = \Pi ( t - 1); h(t) = r(t)$ \\*
\begin{align*}
	y(t) = x(t) * h(t) = \int_{-\infty}^{\infty} r(\tau) \Pi (t - \tau - 1) \, d\tau \\
	= \int_{(t - 1) - \frac{1}{2}}^{(t - 1) + \frac{1}{2}}r(\tau) d\tau
	= \int_{t - \frac{3}{2}}^{t - \frac{1}{2}}r(\tau) d\tau \\
           = \left\{
  		\begin{array}{lr}
		    \int_{t - \frac{3}{2}}^{t - \frac{1}{2}}\tau d\tau & : t\geq \frac{3}{2} \\
		    \int_{0}^{t - \frac{1}{2}}\tau d\tau & : t \in [\frac{1}{2}, \frac{3}{2})\\
	              0 & : t < \frac{1}{2}
		  \end{array}
	\right. \\
	=
	\boxed{
          	\left\{
  		\begin{array}{lr}
		    \frac{(t - \frac{1}{2})^2 - (t - \frac{3}{2})^2}{2} & : t \geq \frac{3}{2} \\
		    \frac{(t - \frac{1}{2})^2}{2} & : t \in [\frac{1}{2}, \frac{3}{2})\\
	              0 & : t < \frac{1}{2}
		  \end{array}
	\right.
	}
\end{align*}
\textbf{b.}
$x(t) = e^{-t}u(t); h(t) = \Pi (t - \frac{1}{2})$\\*
\begin{align*}
	y(t) = x(t) * h(t) = \int_{-\infty}^{\infty} e^{-\tau}u(\tau) \Pi (t - \tau - \frac{1}{2}) \, d\tau \\
	= \int_{(t - \frac{1}{2}) - \frac{1}{2}}^{(t - \frac{1}{2}) + \frac{1}{2}}e^\tau u(\tau) d\tau
	= \int_{t - 1}^{t}e^\tau u(\tau)(\tau) d\tau \\
           = \left\{
  		\begin{array}{lr}
		    \int_{t}^{t - 1}e^\tau  d\tau & : t\geq 1 \\
		    \int_{0}^{t}e^\tau d\tau & : t \in [0, 1)\\
	              0 & : t < 0
		  \end{array}
	\right. \\	
	=
	\boxed{
	\begin{array}{lr}
		    e^t - e^{t - 1} & : t\geq 1 \\
		    e^t & : t \in [0, 1)\\
	              0 & : t < 0
	 \end{array}
	}
\end{align*}
\textbf{c.}
$x(t) = \sum_{n = -\infty}^{\infty} \delta(t - \frac{1}{2} - n) ; h(t) = \Pi(t)Sin(2\pi t)$\\*
\begin{align*}
	y(t) = x(t) * h(t) = \sum_{n = -\infty}^{\infty} \delta(t - \frac{1}{2} - n) * h(t)	= 
	\sum_{n = -\infty}^{\infty}h(t - \frac{1}{2} - n) = \\
	\sum_{n = -\infty}^{\infty} \Pi(t - \frac{1}{2} - n)Sin(2\pi (t - \frac{1}{2} - n)) = 
	\sum_{n = -\infty}^{\infty} \Pi(t - \frac{1}{2} - n)Sin(2\pi t - \pi - 2 n \pi ) = \\
	-\sum_{n = -\infty}^{\infty} \Pi(t - \frac{1}{2} - n)Sin(2\pi t) = \\
	\boxed{Sin(2\pi t)}
\end{align*}

\end{document}




